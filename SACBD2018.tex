%%%% Proceedings format for most of ACM conferences (with the exceptions listed below) and all ICPS volumes.
\documentclass[sigconf,final]{acmart}
%%%% As of March 2017, [siggraph] is no longer used. Please use sigconf (above) for SIGGRAPH conferences.

%%%% Proceedings format for SIGPLAN conferences 
% \documentclass[sigplan, anonymous, review]{acmart}

%%%% Proceedings format for SIGCHI conferences
% \documentclass[sigchi, review]{acmart}

%%%% To use the SIGCHI extended abstract template, please visit
% https://www.overleaf.com/read/zzzfqvkmrfzn


\usepackage{booktabs} % For formal tables
\usepackage{graphicx}

% Copyright
%\setcopyright{none}
%\setcopyright{acmcopyright}
%\setcopyright{acmlicensed}
\setcopyright{rightsretained}
%\setcopyright{usgov}
%\setcopyright{usgovmixed}
%\setcopyright{cagov}
%\setcopyright{cagovmixed}


% DOI
\acmDOI{10.475/123_4}

% ISBN
\acmISBN{123-4567-24-567/08/06}

%Conference
\acmConference[SACBD]{1st Software Architecture Challenges in Big Data}{September 2018}{Madrid, Spain}
\acmYear{2018}
\copyrightyear{2018}


\acmArticle{4}
\acmPrice{15.00}

% These commands are optional
%\acmBooktitle{Transactions of the ACM Woodstock conference}
\editor{Jennifer B. Sartor}
\editor{Theo D'Hondt}
\editor{Wolfgang De Meuter}


\begin{document}
\title{Data Wrangling at Scale}
%Big Data Processing, Integration and Enrichment using Container Orchestration
%\titlenote{Produces the permission block, and copyright information}
\subtitle{The experience of EW-Shopp}
%\subtitlenote{The full version of the author's guide is available \texttt{acmart.pdf} document}


\author{Nikolay Nikolov}
%\authornote{Dr.~Trovato insisted his name be first.}
%\orcid{1234-5678-9012}
\affiliation{%
  \institution{SINTEF}
  \streetaddress{Forskningsveien 1}
  \city{Oslo}
  \country{Norway}
  \postcode{0182}
}
\email{nikolay.nikolov@sintef.no}

\author{Michele Ciavotta}
%\authornote{The secretary disavows any knowledge of this author's actions.}
\affiliation{%
  \institution{University of Milan-Bicocca}
  \streetaddress{DISCo, Viale Sarca 336/14}
  \city{Milan}
  \country{Italy}
  \postcode{20126}
}
\email{michele.ciavotta@unimib.it}

\author{Flavio De Paoli}
%\authornote{This author is the one who did all the really hard work.}
\affiliation{%
  \institution{University of Milan-Bicocca}
  \streetaddress{DISCo, Viale Sarca 336/14}
  \city{Milan}
  \country{Italy}
  \postcode{20126}
  }
\email{flavio.depaoli@unimib.it}



% The default list of authors is too long for headers.
\renewcommand{\shortauthors}{N. Nikolov et al.}


\begin{abstract}
This paper presents a subsystem of a comprehensive platform dedicated to data transformation, linking and extension of large data sets. Furthermore, we detail and discuss both the main requirements that have led to the design and development of the platform, and the devised approach, which is a direct outcome of the requirement elicitation and discussion phase. In particular, the platform supports both design and run time aspects of the data transformation process, which is reflected in the architecture. 
Some initial tests have been carried out on a prototype implementation of our architecture on data sets of \textasciitilde 1TB featuring promising performance. 
\end{abstract}

%
% The code below should be generated by the tool at
% http://dl.acm.org/ccs.cfm
% Please copy and paste the code instead of the example below.
%
\begin{CCSXML}
<ccs2012>
  <concept>
    <concept_id>10010520.10010521.10010537</concept_id>
    <concept_desc>Computer systems organization~Distributed architectures</concept_desc>
    <concept_significance>500</concept_significance>
  </concept>
  <concept>
    <concept_id>10002951.10002952.10003219.10003215</concept_id>
    <concept_desc>Information systems~Extraction, transformation and loading</concept_desc>
    <concept_significance>500</concept_significance>
  </concept>
  <concept>
    <concept_id>10002951.10002952.10003219.10003222</concept_id>
    <concept_desc>Information systems~Mediators and data integration</concept_desc>
    <concept_significance>500</concept_significance>
  </concept>
  <concept>
    <concept_id>10002951.10002952.10003219.10003223</concept_id>
    <concept_desc>Information systems~Entity resolution</concept_desc>
    <concept_significance>500</concept_significance>
  </concept>
  <concept>
    <concept_id>10002951.10003227.10003351.10003218</concept_id>
    <concept_desc>Information systems~Data cleaning</concept_desc>
    <concept_significance>500</concept_significance>
  </concept>
  <concept>
    <concept_id>10002951.10003260.10003309.10003315.10003314</concept_id>
    <concept_desc>Information systems~Resource Description Framework (RDF)</concept_desc>
    <concept_significance>500</concept_significance>
  </concept>
</ccs2012>
\end{CCSXML}

\ccsdesc[500]{Computer systems organization~Distributed architectures}
\ccsdesc[500]{Information systems~Extraction, transformation and loading}
\ccsdesc[500]{Information systems~Mediators and data integration}
\ccsdesc[500]{Information systems~Entity resolution}
\ccsdesc[500]{Information systems~Data cleaning}
\ccsdesc[500]{Information systems~Resource Description Framework (RDF)}


\keywords{Big Data Processing, Data Wrangling, Data Integration, Data Enrichment, Data Extension, Linked Data}


\maketitle


\section{Introduction}

EW-Shopp aims to support e-commerce, retail and marketing industries in improving their efficiency and competitiveness through providing the ability to perform predictive and prescriptive analytics over integrated and enriched large data sets through the use of open and flexible solutions. 
This document is intended to be the first step towards the construction of the EW-Shopp ecosystem, a platform that seeks to significantly reduce integration time and to improve the quality of analytics by providing cutting-edge tools and access to data as a service, with particular reference to weather and event information.  


For some years now we have been witnessing the flourishing and reinforcement of Big Data methodologies in academia but especially in industry. 
This success can be partly explained by the promise of Big Data supporters to define and deliver more accurate and efficient data-driven decision-making processes.
This is not the only asset, though, but it is what we are most interested in highlighting within the EW-Shopp project. 

The term "Data Analytics" is too often used to define a generic data driven decision process and therefore covers heterogeneous activities such as cleaning, linking, enrichment/extension, training and application of analytic models, up to business intelligence and visualization. 
In this document, we propose a platform from its components and their relationships. We present a reference architecture obtained from a rigorous process of requirements elicitation that took into account both literature and best practices as well as the needs expressed by involved stakeholders. 
In that particular document, not only the functional requirements associated with business cases (the implementation and reporting of which in marketed services is one of the core aspects of the project) have been collected and refined, but also the development guidelines based on lean methodology [2] have been defined. In particular, the build-measure-learn feedback loop is also presented. 
In a nutshell, the methodology envisions for each pilot a starting phase with Minimum Viable Product (MVP) meant to reduce the time-to-market. The MVP is going to be enriched as the project progresses and the vision matures. Pilots are therefore seen as playgrounds for experimenting approaches and technologies that eventually will be included in the final marketed services. 


At the time this document is written, we have designed a unified, complete and well-founded reference architecture. The declared objective is to provide a point of reference that will remain as steady as possible for the entire duration of the project and that can thus lead its evolution on the basis of a solid and agreed basis. 

To do this, as we will see, we started from the functional requirements trying to read in them the answers to the architectural questions that we asked ourselves.  It was immediately evident, however, that these requirements were not sufficient to define a modern, efficient, responsive and scalable platform. As a consortium, we have therefore decided to undertake an in-depth study of the problems that we want to tackle. Some results of such teamwork are detailed here.  

With the work the consortium has done on component design, the processes and outcomes of which are described in this document, we aim to achieve the following two objectives:
\begin{itemize}
    \item Define a reference architecture. This deliverable provides a reference architecture that aims at being general and flexible enough to be successful applied in the pilots. The architecture describes the core components and their relationships in terms of data flow. Moreover, it aims at setting a common language among the partners. To make this possible, we have referred to the pilot descriptions, the partners’ experience, and the best practices of the field of Big Data. Furthermore, the effort required to define precisely the pilots, especially in terms of processes, data flow and workflow, has spawned awareness in the consortium members about the complexity of the problem and the need of a common reference. Finally, it is important to note that defining a reference architecture does not conflict with the lean methodology as we describe the components according to their general functionality, without imposing specific solutions.
    \item Propose an initial implementation of the platform. This deliverable also provides an early stage implementation of the reference architecture.
\end{itemize}


This work is structured as follows. 
In Section \ref{sec:architecture} we present the reference architectures for Big Data as well as the main solutions for data wrangling, data analytics, business intelligence and reporting. 
The requirements that have guided the definition of our reference architecture are presented and discussed in Section 3. 
Finally, Section~\ref{sec:conclusions} concludes the document. 









\section{Approach}\label{sec:approach}
We propose a reference architecture specially designed to handle Big Data workloads. 
In order for this to be possible, we have carried out a thorough requirement gathering phase that takes into account the best practices in creating Big Data architectures and the expectations of the business partners involved in the project. 
%As a result, we were able not only to define the platform architecture, but also to set up a unified common language that, we believe, will have clear benefits in the next stages of the project.


The EW-Shopp system aims to support e-commerce, retail and marketing industries in improving their efficiency and competitiveness through providing the ability to perform predictive and prescriptive analytics over integrated and enriched large data sets using open and flexible solutions. In addition, these tools must provide a responsive graphical user interface to guide the user in designing data transformations. 
These observations together with the confidentiality requirement lead us to believe that the data transformation design phase has to be carried out on a small and possibly anonymized subset of the initial data. For this reason, we have introduced into the platform a specialized component, named Sampler, whose task is to properly generate this data subset. 

The idea of reducing the size of the data set to be able to handle it more easily is not new \cite{XXX}[13][14][15][16][17]
and is commonly referred to as Dimension Reduction or Big Data Reduction. The approach is rather simple in its general lines; it consists in reducing the size of the data set by identifying a possible compact representation of it. In this way, the data transformation and data analytics operations can be designed and tested on a smaller set than the original data, which should ensure greater responsiveness and efficiency for the applications involved without negatively affecting accuracy.
\textbf{Figure 6} graphically illustrates how the reduction approach is implemented within the architecture. First of all, we point out that only the Data Wrangler and the Data Analyzer are affected by this methodology as the data reporter will visualize and inspect data of small dimensions which were obtained as a result of the Data Analyzer results. 
As far as the Data Wrangler and Data Analyzer are concerned, the operations associated with the Dimension Reduction Approach are the following:

\begin{enumerate}
    \item Creating a reduced data set (called Sample in the diagram). Such sample may depend on the particular operation to be carried out (preparation or analytics)
    \item 	The user operates the application working on the sample. In the particular case of the Data Wrangler, it also receives in input a collection of recommendations to guide the user in the process of table annotation.
    \item The application generates a machine-readable description of the user's operations
    \item The model (of transformation or analytics) is executed on the initial data set by the Big Data run time component.
\end{enumerate}

It is important to note that this choice does not reduce the applicability and generality of the presented solution as, where it is possible (e. g. in cases where the data to be transformed is manageable and does not require anonymization), the sampling component can be excluded. In this mode, the user is free to work directly against the original data set. 
\section{Reference Architecture}

\subsection{Data Wrangler}

\subsection{Big Data Back end}

%\section{Early evaluation}\label{sec:evaluation}

- jot data
- some information about timing

\section{Conclusions}\label{sec:conclusions}
%The EW-Shopp ecosystem aims to support e-commerce, Retail and Marketing industries in improving their efficiency and competitiveness through providing the ability to perform predictive and prescriptive analytics over integrated and enriched large data sets using open and flexible solutions. 
In this paper, we have outlined the architecture of a platform designed to transform, link and extend massive data sets. The architecture features three logical tiers with distinct tasks, providing support for both design-time and run-time (definition of data clean-up and transformation, definition of data flows and their execution). As of the time of the writing of this paper, the EW-Shopp platform is under active development. A prototype of the platform has been deployed and tested for project business cases. Moreover, preliminary experiments have been carried out where the prototype has been used to successfully execute a data flow to continuously clean-up, transform, enrich sample data in the magnitude of \textasciitilde 1TB data demonstrating the viability of the approach and showing promising performance.

%design component, which provides the user with advanced tools to design a transformation and enrichment pipeline working  on an reduced version of the original data set, and a run time system, which realizes the full-scale execution of the resulting pipeline. 



\begin{acks}
The work in this paper is partly supported by the H2020 project EW-Shopp (Grant number: 732590).
\end{acks}


\bibliographystyle{ACM-Reference-Format}
\bibliography{references}

\end{document}
