\section{Conclusions}
The EW-Shopp ecosystem aims to support e-commerce, Retail and Marketing industries in improving their efficiency and competitiveness through providing the ability to perform predictive and prescriptive analytics over integrated and enriched large data sets using open and flexible solutions. 
With this document, we have tried to pursue several objectives at once. Mainly, we have outlined a reference architecture specifically designed to handle large amounts of data. This architecture is based on the principle of the data set Dimension Reduction to work. We have tried to explain to the reader how, by appropriately reducing the size of the data set, it is possible to design a data enrichment process that is both precise and manageable. The same principle underpins the process of defining analytics algorithms. The proposed architecture naively supports this approach but can also work directly on business data as long as it is not too large.
Our second objective was to raise awareness among consortium members about the inherent difficulties associated with the project and the possible solutions that could be adopted. We have therefore attempted to create a unified common language that, we believe, will have clear benefits in the next stages of the project.
Finally, we have tried to propose an initial implementation of the architecture by identifying one or more tools for each component of the reference architecture. In particular, at this stage we have embraced the lean development approach already introduced in Deliverable D4.1. In our case this has meant that only the centerpiece of the platform, namely the tools already identified and on which an agreement exists among the consortium members, is presented as stable and detailed. The rest of the platform, on the other hand, is constituted by tools and/or technologies that at this moment seem to be good candidates but that can be replaced as the project develops. In addition, business cases can evolve independently and have individual requirements. This could lead to the adoption of different tools to cover the same functionalities. Nevertheless, we believe that the reference architecture and its implementation, however incomplete it may be, will be able to guide the development process that will take place in the coming months.
