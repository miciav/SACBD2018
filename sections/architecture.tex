\section{Wrangling Architecture}

We propose a reference architecture specially designed to handle Big Data workloads. 
In order for this to be possible, we have carried out a thorough requirement gathering phase that takes into account the best practices in creating Big Data architectures and the expectations of the partners involved in business cases. As a result, we were able not only to define the platform architecture, but also to set up a unified common language that, we believe, will have clear benefits in the next stages of the project.
In addition, we propose an initial reference implementation by selecting one or more solutions for each reference component. In particular, embracing the lean development approach, we detail only the core of the platform, namely the tools on which a strong convergence exists among the consortium members (the minimum viable product or MVP). The rest of the platform is constituted by suggestion of tools and/or technologies that at this moment seem to be good candidates but that may be replaced as the project develops. 


\subsection{Data Wrangler}
descrizione di come funziona.
messaggio, non un tool per esperti tecnologi ma esperti di dominio si.

\subsection{Dataspace and Big Data Back end}
